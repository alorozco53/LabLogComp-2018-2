\documentclass[paper=letter, fontsize=12pt]{scrartcl}

\usepackage[T1]{fontenc}
\usepackage[utf8]{inputenc}
\usepackage[english, spanish]{babel}
\usepackage{hyperref}
\usepackage{amsfonts,amsthm}
\usepackage{amsmath}
\usepackage{listings}
\usepackage[dvipsnames]{xcolor}
\usepackage{sectsty}
\usepackage{dirtree}
\usepackage{enumitem}

\lstdefinestyle{mystyle}{
    backgroundcolor=\color{backcolour},
    commentstyle=\color{codegreen},
    keywordstyle=\color{magenta},
    numberstyle=\tiny\color{codegray},
    stringstyle=\color{codepurple},
    basicstyle=\footnotesize,
    breakatwhitespace=false,
    breaklines=true,
    captionpos=b,
    keepspaces=true,
    numbers=left,
    numbersep=5pt,
    showspaces=false,
    showstringspaces=false,
    showtabs=false,
    tabsize=2
}

\setlength{\DTbaselineskip}{20pt}
\DTsetlength{1em}{3em}{0.1em}{1pt}{4pt}

\allsectionsfont{\raggedright\large \textit\normalfont\scshape\emph}

\title{Práctica 1}

\subtitle{
  Lógica Computacional, 2018-2\\
  Facultad de Ciencias, UNAM
}

\author{
  \normalsize
  Noé Salomón Hernández Sánchez\\
  \normalsize
  \texttt{\href{mailto:no.hernan@gmail.com}{no.hernan@gmail.com}}
  \and
  \normalsize
  María del Carmen Sánchez Almanza\\
  \normalsize
  \texttt{\href{mailto:carmensanchez@ciencias.unam.mx}{carmensanchez@ciencias.unam.mx}}
  \and
  \normalsize
  Albert Manuel Orozco Camacho\\
  \normalsize
  \texttt{\href{mailto:alorozco53@ciencias.unam.mx}{alorozco53@ciencias.unam.mx}}
}

\date{\today}

\begin{document}

\maketitle

\section{Objetivo}

\noindent
Que el alumno utilice los conocimientos básicos de Haskell, para implementar el comportamiento\
del cálculo proposicional.

\section{Ejercicios}

\noindent
El esqueleto de código del ejercicio semanal se encuentra en \url{https://github.com/alorozco53/LabLogComp-2018-2/tree/pract1}.\
Considere el tipo de datos
\begin{lstlisting}[language=Haskell]
  data Prop = VarP String
            | TTrue
            | FFalse
            | Neg Prop
            | Disj Prop Prop
            | Conj Prop Prop
            | Imp Prop Prop
            | Equiv Prop Prop
\end{lstlisting}
el cual codifica todas y cada una de las fórmulas de la lógica proposicional.

\begin{enumerate}
\item Implemente una función que imprima una fórmula de la lógica proposicional\
  mediante operadores infijos. Por ejemplo:
    \begin{lstlisting}[language=Haskell]
    *Practica1> Equiv (VarP "p") (Conj (TTrue) (VarP "q"))
    ("p") <-> ((True) ^ ("q"))
    \end{lstlisting}
\end{enumerate}

\noindent
Considere el siguiente tipo \textit{alias} para sustituciones:
\begin{lstlisting}[language=Haskell]
  type Sub = String -> Prop
\end{lstlisting}
Entonces, una sustitución se definirá mediante una función que va de variables a proposiciones.\
Por ejemplo, la sustitución $[p := q \leftrightarrow \perp,\ q := True]$ se especificaría mediante
\begin{lstlisting}[language=Haskell]
sub1 :: Sub
sub1 "p" = Equiv (VarP "q") FFalse
sub1 "q" = TTrue
sub1 other = VarP other
\end{lstlisting}
Obsérvese, entonces, que el último caso se usa para cualquier otra variable que no esté definida\
en la sustitución (i.e., función \emph{identidad}).

\begin{enumerate}[resume]
\item Escriba una función \verb+substitute+ que, dada una proposición $\Phi$ y una sustitución $s$\
  (de tipo \verb+Sub+), aplique $s$ en $\Phi$.
\end{enumerate}

\noindent
Considere el siguiente tipo \textit{alias} para estados (interpretaciones):
\begin{lstlisting}[language=Haskell]
  type State = [String]
\end{lstlisting}
Para una fórmula $\Phi$ de la lógica proposicional, asumimos que cualquier variable $v$ presente\
en $\Phi$ posee una interpretación $\mathcal{I}(v) = 1$ \emph{si y sólo si}\
$v$ está dentro de la lista de tipo \verb+State+ dada.

\begin{enumerate}[resume]
\item Proponga una función \verb+interp+ que, dada una fórmula de tipo \verb+Prop+\
  y una interpretación de tipo \verb+State+, devuelva la interpretación de la fórmula\
  en tipo \verb+Bool+ de Haskell.
\item Elabore una función \verb+model+ que decida si el estado dado \textbf{es un modelo}\
  para una fórmula dada.
\item Escriba una función \verb+vars+ que devuelva todas las variables contenidas\
  en la fórmula proposicional dada.
\item Dé una función \verb+powerList+ que devuelva la \emph{lista potencia} de una lista\
  dada. Por ejemplo, dada la lista \verb+[1, 2]+, su lista potencia sería\
  \verb+[[], [1], [2], [1, 2]]+. \emph{Sugerencias}:
  \begin{itemize}
  \item Sea $A$ un conjunto finito de tamaño $n$. Demuestre (con inducción matemática)\
    que el conjunto potencia de $A$ posee $2^n$ elementos.
  \item Implemente la demostración anterior, usando una lista por comprensión en el caso recursivo.
  \end{itemize}
\item Escriba una función \verb+tautology+ que determine si una fórmula dada es una \emph{tautología}.\
   \emph{Sugerencia}:
   \begin{itemize}
   \item Utilice la función \verb+powerList+ para calcular todas las interpretaciones de las variables\
     de una fórmula dada.
   \end{itemize}
\item Escriba una función \verb+equivProp+ que determine si dos fórmulas dadas\
  son \emph{lógicamente equivalentes}. \emph{Sugerencia}:
  \begin{itemize}
  \item Utilice el teorema visto en clase que establece que dos fórmulas son lógicamente\
    equivalentes \emph{si y sólo si} cierta fórmula es una tautología. ¿A qué fórmula nos referimos?
  \end{itemize}
\item Escriba una función \verb+logicConsequence+ que, dada una lista de fórmulas $\Gamma$, y una\
  proposición $\Psi$, determine si $\Gamma \models \Psi$.
\end{enumerate}

\section{Entrega}

\noindent
La fecha de entrega es el próximo \textbf{sábado 3 de marzo de 2018} por la plataforma\
de \emph{Google Classroom} del curso y siguiendo los lineamientos del laboratorio.

\end{document}
